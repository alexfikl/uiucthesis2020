% \iffalse meta-comment
% Package and Class "uiucthesis2020" for use with LaTeX2e.
% \fi
% \iffalse
%<*driver>
\documentclass{ltxdoc}
\def\fileversion{v2.30b}
\def\filedate{2020/02/20}

\usepackage{hyperref}

\begin{document}
\title{
    The \textsf{uiucthesis2020} class
    \thanks{This file has version number \fileversion, last revised \filedate.}}
\author{}
\date{\filedate}
\maketitle
\DocInput{uiucthesis2020.dtx}
\end{document}
%</driver>
% \fi
%
% \MakeShortVerb{\|}
%
% \def\pkg#1{\textsf{#1}}
% \def\env#1{\textsf{#1}}
%
% \begin{abstract}
% Load the \pkg{uiucthesis2020} class for use with \LaTeX2e
% to produce a document that should conform to the format described by
% the Graduate College requirements~\cite{thesisweb}.
% \end{abstract}
%
% \section{The User Interface}
%
% This section describes how to use the \pkg{uiucthesis2020} class to
% produce a thesis satisfying the format requirements of the
% Grad College at UIUC. Note that this is an unofficial template so, while
% we try to make it as close as possible to the requirements, there is no
% guarantee it will be accepted as is.
%
% Any modifications, issues or bugs can be contributed back to the project
% at~\cite{github}.
%
% \subsection{Using \pkg{uiucthesis2020}}
%
% To write a thesis, you load the UIUC thesis definitions
% by loading the \pkg{uiucthesis2020} class at the beginning of
% your \LaTeX\ document with the |\documentclass| command.
% For example,
%
% \begin{quote} \begin{verbatim}
% \documentclass[edeposit,drafthesis]{uiucthesis2020}
% \usepackage{blindtext}
% \end{verbatim} \end{quote}
%
% \iffalse
%<*example>
\documentclass[11pt,edeposit,draftthesis]{uiucthesis2020}

\usepackage{blindtext}

\usepackage{fixmath}
\usepackage{amsmath}
\usepackage{amsthm}
\usepackage{amssymb}
\usepackage{stmaryrd}

\usepackage{xparse}
\usepackage{hyperref}
\usepackage{cleveref}
\hypersetup{
    colorlinks=true,
    urlcolor=blue,
    citecolor=black,
    linkcolor=black
}

\usepackage[shortlabels]{enumitem}
\usepackage{booktabs}

\usepackage[sc]{mathpazo}
\usepackage[activate={true,nocompatibility}, % activate protrusion and font expansion
            final,              % enable microtype, use draft to disable
            tracking=true,
            kerning=true,       % optimise interactions between characters
            spacing=true,       % more uniform spacing between words
            factor=1100,        % more protrusion
            stretch=10,         % smaller values (default 20, 20) to avoid blurring
            shrink=10]{microtype}
\microtypecontext{spacing=nonfrench}
\SetTracking{encoding={*}, shape=sc}{40}

%</example>
% \fi
%
% The \pkg{uiucthesis2020} class provides a number of options. These are
% meant to help with reviewing draft versions of the manuscript. Since the
% requirements from~\cite{thesisweb} are fairly strict, there are no options
% to modify the format.
%
% \DescribeMacro{[draftthesis]}
% The |[draftthesis]| option causes each page to have a header
% proclaiming the document to be a draft copy along with the
% current time and date. It also omits the copyright page and adds line
% numbers to the main matter and appendices for easy reference during review.
%
% \DescribeMacro{[edeposit]}
% Use the |[edeposit]| option if you are depositing your thesis electronically.
% The differences include a slightly changed page numbering (due to the fact
% that the committee approval form is not included) and the use of a
% |oneside| format for the pages.
%
% \DescribeMacro{[doublespacing]}
% Use the |[doublespacing]| option to turn on double-spaced lines. By default
% the lines are spaced at 1.5 instead.
%
% \subsection{The Title Page}
%
% The |\maketitle| command is redefined so that it creates a title page with
% the correct format for a thesis at UIUC.
%
% \DescribeMacro{\phdthesis}
% \DescribeMacro{\otherdoctorate}
% \DescribeMacro{\msthesis}
% \DescribeMacro{\othermasters}
% Use the |\phdthesis| or |\msthesis| to set the correct thesis type.
% If your thesis isn not for a ``Ph. D.'' or ``M. S.'', you can specify
% your degree with either
%
% \begin{center}
% \begin{tabular}{l}
% |\otherdoctorate{|\meta{degree name}|}{|\meta{abbreviation}|}| \\
% |\othermasters{|\meta{degree name}|}{|\meta{abbreviation}|}|.
% \end{tabular}
% \end{center}
%
% For example, specifying |\phdthesis| is equivalent to giving the command
%
% \begin{center}
% |\otherdoctorate{Doctor of Philosophy}{Ph.D.}|.
% \end{center}
%
% \DescribeMacro{\department}
% \DescribeMacro{\college}
% Set your department with |\department{|\meta{department}|}|. This defines
% the field your degree will be in, so leave out ``Department of''.
% Define your college with |\college{|\meta{college}|}|.
% The default is college is ``Graduate College'' and should probably not be
% changed.
%
% \DescribeMacro{\schools}
% Use |\schools{|\meta{school list}|}| to list the previous degrees
% you have received and the schools that you received them from.
% Separate multiple degrees with |\\|.
%
% \DescribeMacro{\degreeyear}
% Use |\degreeyear{|\meta{year}|}| to define the year in which
% you will receive your degree.  The default is the current year.
%
% \DescribeMacro{\advisor}
% \DescribeMacro{\adviser}
% Use |\advisor{|\meta{advisor name}|}| or |\adviser{|\meta{advisor name}|}|
% to specify the name of your advisor.
%
% \DescribeMacro{\committee}
% Use |\committee{|\meta{committee members}|}| to specify the members
% of your committee and their titles as you want them to appear on the
% title page. Separate members with |\\|. To respect the graduate college
% guidelines, you must use the full title of each committee members.
% The committee chair should appear first with the designation ``, Chair''.
% Your thesis adviser should appear second with the title
% ``, Director of Research''. See the graduate college website for details.
%
% \DescribeMacro{\volume}
% The |\volume| macro provides nominal support for very long theses that must
% be broken up into multiple volumes. Use |\volume{|\meta{number}|}|
% to specify the volume number (a single arabic numeral). All this macro
% does is place the word VOLUME with the number you specify on the title
% page. You have to take care of what appears in each volume. The easiest
% way to do this is to create two separate source files, one for each
% volume.
%
% Here's how to produce an example similar to that in
%
% \begin{quote} \begin{verbatim}
\begin{document}

\title{Coffee Consumption of Graduate Students Trying to Finish Dissertations}
\author{Juan Valdez}
\department{Food Science}
\schools{
    B. A., University of Columbia, 1981 \\
    A. M., University of Illinois at Urbana-Champaign, 1986}
\phdthesis
\advisor{Java Jack}
\degreeyear{1994}
\committee{
    Professor Prof Uno, Chair \\
    Professor Prof Dos, Director of Research \\
    Assistant Professor Prof Tres \\
    Adjunct Professor Prof Quatro}
\maketitle
% \end{verbatim} \end{quote}
%
% \subsection{Front Matter}
%
% \DescribeMacro{\frontmatter}
% Typically, a thesis might have an Abstract, a Dedication, some
% Acknowledgments, and a Preface before the Table of Contents.
% Use the |\frontmatter| command to start this preliminary section
% of the thesis.
% The |\frontmatter| command sets the page number of the next page
% to roman numeral iii (or ii if the |[edeposit]| option is used).
% (The title page is page i, and the certificate
% of committee approval, the ``red-bordered form,'' is page ii.)
%
% \DescribeEnv{abstract}
% The abstract should appear in the \env{abstract} environment. Normally,
% this just produces another chapter with |\chapter*{\abstractname}|,
% where |\abstractname| is ``Abstract'' (see User Customization below),
%
% \DescribeEnv{dedication}
% A dedication page can be printed with the \env{dedication} environment.
% This produces a separate page with the dedication centered horizontally
% and vertically, with the text in italics.
%
% After this front matter comes the Table of Contents,
% List of Tables, List of Figures, etc.  Use the standard \LaTeX\
% commands |\tableofcontents|, |\listoftables|, |\listoffigures|, etc.,
% to generate them. In the \pkg{uiucthesis2020} format these lists are all
% single spaced.
%
% \DescribeEnv{symbollist}
% \DescribeEnv{symbollist*}
% Optionally, these tables can be followed by a List of Abbreviations and/or
% List of Symbols. Introduce these with the |\chapter| command. To aid in
% making these lists, the \env{symbollist} and \env{symbollist*} environments are
% defined in \pkg{uiucthesis2020}. These environments produce a two-column list
% as illustrated below. By default the left column is 1 inch wide but can
% be specified with an optional argument. In the starred environment, the left
% column is left-justified, otherwise it is centered. See the example below.
%
% Here's an example of what the front matter of a typical
% thesis looks like.  First comes the Abstract and the Dedication, both of
% which are optional.
% \begin{quote} \begin{verbatim}
\begin{frontmatter}

\begin{abstract}
This is a comprehensive study of caffeine consumption by graduate
students at the University of Illinois who are in the very final
stages of completing their doctoral degrees. A study group of six
hundred doctoral students\ldots.
\end{abstract}

%% Create a dedication in italics with no heading, centered vertically
%% on the page.
\begin{dedication}
To Father and Mother.
\end{dedication}

%% Create an Acknowledgements page, many departments require you to
%% include funding support in this.
\chapter*{Acknowledgments}

This project would not have been possible without the support of
many people. Many thanks to my adviser, Lawrence T. Strongarm, who
read my numerous revisions and helped make some sense of the
confusion. Also thanks to my committee members, Reginald Bottoms,
Karin Vegas, and Cindy Willy, who offered guidance and support.
Thanks to the University of Illinois Graduate College for awarding
me a Dissertation Completion Fellowship, providing me with the
financial means to complete this project. And finally, thanks to
my husband, parents, and numerous friends who endured this long
process with me, always offering support and love.

%% The thesis format requires the Table of Contents to come
%% before any other major sections, all of these sections after
%% the Table of Contents must be listed therein (i.e., use \chapter,
%% not \chapter*).  Common sections to have between the Table of
%% Contents and the main text are:
%%
%% List of Tables
%% List of Figures
%% List Symbols and/or Abbreviations
%% etc.

\tableofcontents
\listoftables
\listoffigures
% \end{verbatim} \end{quote}
%
% If you want a List of Symbols or Abbreviations, you can do so as follows:
%
% \begin{quote} \begin{verbatim}
%% Create a List of Abbreviations. The left column
%% is 1 inch wide and left-justified
\chapter{List of Abbreviations}

\begin{symbollist*}
\item[CA] Caffeine Addict.
\item[CD] Coffee Drinker.
\end{symbollist*}

%% Create a List of Symbols. The left column
%% is 0.7 inch wide and centered
\chapter{List of Symbols}

\begin{symbollist}[0.7in]
\item[$\tau$] Time taken to drink one cup of coffee.
\item[$\mu$g] Micrograms (of caffeine, generally).
\end{symbollist}

\end{frontmatter}
% \end{verbatim} \end{quote}
%
% \subsection{Main Matter}
%
% \DescribeMacro{\mainmatter}
% Begin the main body of your thesis with the |\mainmatter| command.
% It resets the page number to arabic numeral 1. You can now use any of the
% commands defined by the the |srcbook| document class to write your thesis.
%
% \begin{quote} \begin{verbatim}
\begin{mainmatter}

%% Sample chapter to test margins
\chapter{Introduction}

\Blindtext[6]

\chapter{Methods}

How does all this relate to coffee? We direct the reader
to~\cite{Walker2015} and~\cite{Hager2006}.

\Blindtext[6]

\chapter{Results}

\Blindtext[6]

\chapter{Conclusions}

We conclude that graduate students like coffee.

\end{mainmatter}
% \end{verbatim} \end{quote}
%
% \subsection{Back Matter}
%
% \DescribeMacro{\backmatter}
% The last few chapters in your thesis should not have chapter
% numbers, but should be listed in the Table of Contents.
% These chapters include the Bibliography and the Index. \LaTeX's
% |\backmatter| command accomplishes this.
%
% \DescribeMacro{\bibliography}
% Use the standard \LaTeX\ bibliography commands to
% create your bibliography.
%
% \begin{quote} \begin{verbatim}
\begin{backmatter}

\bibliographystyle{apalike}
\bibliography{references}

\end{backmatter}
% \end{verbatim} \end{quote}
%
% \subsection{Reference Matter}
%
% \DescribeMacro{\appendix}
% To switch from the body of your thesis to the reference material
% at the end, you should use the standard \LaTeX\ |\appendix| command.
% In \pkg{uiucthesis2020}, there is also a starred version of this command
% that eliminates the lettering of the appendices (use if you have
% a single appendix).
%
% \begin{quote} \begin{verbatim}
\appendix
\chapter{My Appendix}
\Blindtext[6]

\end{document}
% \end{verbatim} \end{quote}
%
% \section{User Customization}
%
% \DescribeMacro{\draftheader}
% If you don't like the header that the the |[draftthesis]| option
% creates, you can redefine the |\draftheader| command to any text.
%
% \DescribeMacro{\nocopyrightpage}
% Unless the |[draftthesis]| option is used, a page with the copyright notice
% is printed before the title page. If you don't want this page to
% appear, even in the final version, put the |\nocopyrightpage| macro
% somewhere in the preamble.
%
% \DescribeMacro{\bibname}
% |\bibname| is a standard \LaTeX\ macro that contains the title of the reference
% section at the end of your thesis; ``References'' by default. Use
% |\renewcommand| to redefine it in the preamble.
%
% \DescribeMacro{\abstractname}
% Like above, but for the abstract. By default, ``Abstract''.
%
% \StopEventually{%
% \begin{thebibliography}{9}
%   \bibitem{thesisweb}
%       \emph{Graduate College Thesis Requirements}.
%       \newblock |https://grad.illinois.edu/thesis/format|
%
%   \bibitem{github}
%       \emph{UIUC Thesis Github}.
%       \newblock |https://github.com/alexfikl/uiucthesis2020|
% \end{thebibliography}
% }
% \Finale
\endinput

%% vim: nospell:filetype=tex
